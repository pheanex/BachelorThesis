\chapter{Introduction}
\section{Motivation}
\begin{description}
\item[Description of Status Quo]
\item[What is the Problem?]
AutoWDS does not scale/work well
\item[What do we want?]
Make AutoWDS Useful/scalable/work again
\item[Why are we doing this?]
Provide a system to easily deploy of a set of accesspoints for creating a wireless lan infrastructure
\item[How do we want to do/solve this?]
Using multiple Channels
Using Redundant Paths Using a better Network Topology
\item[What problems could arise?]
Finding a good network-topology/CAA tricky
\item[How are we planing on dealing with these problems?]
\item[What (data) do we need to solve the Problem?]
Central Inter-AP SNR
\end{description}

\section{Structure of Thesis}
Description of what's in the chapters.

Below This : examples

\begin{table}[b]
\caption{Tables should look like this (save for the last row). If you know LaTeX better than me, feel free to improve the way of producing these tables.}
\begin{tabularx}{\linewidth}{|l|X|X|}
\hline
\rowcolor{slightgray}
\T Tables	&have gray  &headlines\\
\hline
\cellcolor{slightgray}\T and gray &labels \B&, too.\\
\hline
\cellcolor{slightgray}\T T &and B& are used for spacing\B\\
\hline
\cellcolor{slightgray} without T & and B& the cells are too small\B\\
\hline 
\end{tabularx}
\label{tab:example}
\end{table}

%These are reference and cite examples. See Figure~\ref{fig:example}, Table~\ref{tab:example}, and Listing~\ref{lst:example}. Cite early and often~\cite{exampleentry} is a good rule of thumb.
