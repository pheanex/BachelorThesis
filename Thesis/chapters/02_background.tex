\chapter{Background}
How shallow / extensive should the description be? Wikipedia like or original techpaper of 802.11 like?
\section{Wireless Networks}
  \subsection{WLAN Channel}
    %https://de.wikipedia.org/wiki/Wireless_LAN#Frequenzen\newline
    \begin{description}
    \item[What is a Channel?]
    \item[Overview on useable frequencies]
    \item[Interference]
    \item[Hidden Station Problem]
    \item[CSMA-CD/CA]
    \item[Analogy switch - collision domain]
    \end{description}
  \subsection{Wireless Access-Point}
    %Used to establish wireless connections to devices, like laptops/mobile phones, but also printer and ... 
    %Connecting wired LAN - wireless LAN \newline
    %Mostly provide access to internet\newline
    %Usecases of Accesspoints\newline
    %https://de.wikipedia.org/wiki/Wireless_Access_Point \newline
  \subsection{Wireless Mesh Network}
    %https://de.wikipedia.org/wiki/Ad-hoc-Netz
    %Topology for nodes(devices) in a wireless network.
    %Self healing capabilities.
    %Role of Mesh routers.
    %Redundancy in mesh networks.
    %Autonomy of devices.
    %Infrastructure mode vs p2p mode
    %IEEE 802.11
    %Alterations in our scenario compared to standard mesh\newline
    \cite{Akyildiz2005445}
    \cite{airberry}
    \subsection{Wireless Distribution System}
    %https://en.wikipedia.org/wiki/Wireless_distribution_system
    Description of WDS + AutoWDS \newline
      \begin{description}
       \item[What is a Wireless Distribution System?]
       \item[How does the Lancom AutoWDS differ from the basic WDS definition?]
      \end{description}
\section{Graph theoretic basics}
  %https://de.wikipedia.org/wiki/Graph_%28Graphentheorie%29
  \begin{description}
   \item[How to map the real world scenario to graph theory]
    APs are nodes \newline
    Also Radio-Modules of APs are nodes and are connected to AP-nodes. Those are called Module-Device edges or fake edges.\newline
    Connections between nodes are possible Links, meaning if there is a Edge from AP1-Module1 to AP2-Module2, then AP1-Module1 and AP2-Module2 received each others beacons. Those are called Module-Module edges or real edges. \newline
   \item[How is our Channel assignment related to the problem COLORING?]
    It is realated in the way that we want to assign each Module-Module-Connection a channel/color from a pool of available colors, withouth (if possible) use the same channel/color on neighboring links (meaning Aps that see each other and their traffic could lead to decreased throughput due to interference)
    \item[What else here?]
 \end{description}
