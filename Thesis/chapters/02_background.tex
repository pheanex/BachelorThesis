\chapter{Background}
\section{Wireless Network}
  \subsection{WLAN Channel}
    %https://de.wikipedia.org/wiki/Wireless_LAN#Frequenzen\newline
    What is a Channel?\newline
    Overview on useable frequencies \newline
    Interference\newline
    Hidden Station Problem\newline
    CSMA-CD/CA\newline
    Analogy switch - collision domain \newline
  \subsection{Wireless Access-Point}
    Used to establish wireless connections to devices, like laptops/mobile phones, but also printer and ... 
    Connecting wired LAN - wireless LAN \newline
    Mostly provide access to internet\newline
    Usecases of Accesspoints\newline
    %https://de.wikipedia.org/wiki/Wireless_Access_Point \newline
  \subsection{Wireless Mesh Network}
    %https://de.wikipedia.org/wiki/Ad-hoc-Netz
    Topology for nodes(devices) in a wireless network.
    Self healing capabilities.
    Role of Mesh routers.
    Redundancy in mesh networks.
    Autonomy of devices.
    \cite{Akyildiz2005445}
    Infrastructure mode vs p2p mode
    IEEE 802.11
    \cite{airberry}
    Alterations in our scenario compared to standard mesh\newline
\section{Graph theoretic basics}
  %https://de.wikipedia.org/wiki/Graph_%28Graphentheorie%29
  How to map the real world scenario to graph theory\newline
    APs are nodes \newline
    Also Radio-Modules of APs are nodes and are connected to AP-nodes. Those are called Module-Device edges or fake edges.\newline
    Connections between nodes are possible Links, meaning if there is a Edge from AP1-Module1 to AP2-Module2, then AP1-Module1 and AP2-Module2 received each others beacons. Those are called Module-Module edges or real edges. \newline
  How is our Channel assignment related to the problem COLORING?\newline
    It is realated in the way that we want to assign each Module-Module-Connection a channel/color from a pool of available colors, withouth (if possible) use the same channel/color on neighboring links (meaning Aps that see each other and their traffic could lead to decreased throughput due to interference)
  What else here?
\section{Wireless Distribution System}
  %https://en.wikipedia.org/wiki/Wireless_distribution_system
  Description of WDS + AutoWDS \newline
    What is a Wireless Distribution System? \newline
    How does the Lancom AutoWDS differ from the basic WDS definition? \newline
