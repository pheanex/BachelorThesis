\chapter{Related Work}
  The following four approaches are all solutions that are graph-based and target a centralized and static channel assignment.
  Those are not the only ones on this matter, but the most known and relevant contestants to our problem.
  Note that there are also numerous solutions on distributed and/or dynamic or semi-dynamic environments, which we will not covere here.
  Except the BFS-CA, the prevalent solutions make excessive use of the conflict graph, which according to \cite{overview_caa} has difficulties to model 
  the varying number of radios equipped on accesspoints. Also most of the works solely focus on assigning channels to an existing topology and do not touch 
  the network topology at all, what seems like a handicap as topology control is an essential part in dealing with wireless interference.
  
  \section{\ac{CLICA}}
    The aim of \ac{CLICA} \cite{CLICA} is to minimize interference conflicts for a given unit disk graph preserving the connectivity.
    That means it takes the given network topology-graph as granted and tries to minimize the overall interference by resolving conflicts as much as possible.
    It takes the following parameters as input:
    
    \begin{itemize}
      \item Unit disk graph
      
      \item Number of radios at each node and total number of channels available
      
      \item Interference conflicts in form of a conflict graph
    \end{itemize}

    \begin{figure}
      \centering
      \includegraphics[width=1\columnwidth]{figures/unit-disk-graph}
      \caption{Creation of an unit disk graph from the receive range map of accesspoints}
      \label{fig:unit-disk-graph}
    \end{figure}
    
    \newpage
    
    CLICA's mode of operation is described by \cite{overview_caa} as follows:
    \begin{quote}
      \begin{itemize}
	\item Randomly assign a node \(v\) the highest priority, then assign other
	  nodes priorities decreasing in the order obtained by depth.
	  
	\item While traversing the nodes in the decreasing order of their
	  priorities obtained above, assign channels to the incident links
	  of these nodes. The operation of assigning a channel to a link
	  includes assigning this channel to both a radio at this node and
	  a radio at the neighbor node. Then, the priorities of unvisited
	  nodes are adjusted according to their degree of flexibility, which
	  is the number of channels that a node can choose from without
	  breaking the connectivity preservation. Essentially, the nodes
	  with a lower degree of flexibility will have their priorities
	  increased so that they are visited earlier in the later steps.
	  
	\item When picking a channel in the above step, a node v1 picks
	  a channel for its incident link (\(v_1\) , \(v_2\) ) in a greedy manner: a
	  locally optimal choice is made by selecting the channel that
	  minimizes the maximum link conflict weight among all links
	  that can interfere with link (\(v_1\) , \(v_2\) ). After a channel is assigned
	  to a link, the conflict graph is updated to reflect the new link conflict weights.
      \end{itemize}
    \end{quote}
  
    The reason why we decided not to use this algorithm is that \ac{CLICA} only tries to minimize interference by assigning channels the best way possible for a given
    unit disk graph, which is basically each possible connection. For a highly connected network topology like \ref{fig:graphseen} this would lead to suboptimal results
    as it does not restrict itself to necessary connections and therfore possesses a lot of potential for interference.
    The restriction to use only the best and absolutely neccessary links is a vital part in order to 
    further decrease interference as much as possible for such a topology.
    
  \section{\ac{INSTC}}
    As pointed out by \cite{overview_caa}, INSTC \cite{INSTC} is similar to \ac{CLICA} with a few alterations, 
    which is why we will not go into further detail and rather focus on its differences. 
    They introduce \ac{LCI} which for a link represents the number of links which interfere with this link \cite{overview_caa} and serves as a measure of 
    interference. Additionally they accept \(k\) as input parameter which results in a \(k\)-connected Graph as outcome to make the topology resilient to node failures.
    Although this feature would come in handy for our survival path requirement, it does not match our failing scenario of a single link at a time instead of a whole node outage.
    Using a \textit{k}-(node)-connected graph instead of an \textit{k}-edge-connected graph increases the the number of edges that have to be utilized 
    (since one failing node involves several failing edges). Consequently the resulting network topology has a higher grade of connectivity than it needs to 
    have and therefore conversely affects overall throughput since a higher node connectivity leads to fewer useable different channels. 
    Additional to the weaknesses of \ac{CLICA}, they also require the number of radio modules per accesspoint to be identical, 
    which is not desireable for us as requirement analysis dictates the operability also for heterogenous networks where this
    restriction is not met. The LCI measure introduced in the work by \cite{INSTC} served as a basis for our edgescore calcuation in ~\ref{eq:edgescore}.
    
  \section{\ac{BFS-CA}}
    \ac{BFS-CA} \cite{BFS-CA} extends the common conflict graph with modules, resulting in a \ac{MCG} graph, which more closely reassembles the
    real world setup. This made it easier for them to deal with the different numbers of radios available on each accesspoints. 
    What they do is to let the accesspoints (or mesh routers in their case) sniff the network on regular intervalls and determine a ranking for the channels.
    Those rankings are then sent to their central entity, the \ac{CAS}. This \ac{CAS} in turn derives the \ac{MCG} and assigns each node in this graph 
    a channel in \ac{BFS}-manner by considering the received rankings of the accesspoints. In order not to partition the network by introducing a \ac{CA}, which 
    would disconnect some links by setting certain radios to different channels, they also use one radio on each accesspoint on an overall-common channel.
    This common channel is then used for initial setup, management frames and as a backup if other links break in order to keep the graph connectivity attribute.
    
    Although \ac{BFS-CA}, as others, is still anxious about topology alterations, since on a first glance it might introduce too severe problems like, 
    additional hops for packets, increased interference footprint and higher susceptibility to errors due to longer travel times of packets.
    Nevertheless it is the algorithm which we were inspired by the most and therefore have some ideas in common.
    The interesting features we reused and refined are:
    
    \begin{itemize}
     \item Idea of a ranking system for each possible channel
     
     \item Central computation on a \ac{CAS}, which reassembles our \ac{WLC}
     
     \item Taking foreign sources of interference also into consideration, instead of just internal.
     
    \end{itemize}
    
    Yet, we decided against its implementation for our purposes for the following reasons:
    
    \begin{itemize}
      \item Using all possible channels in the network topology creates to much interference for networks with more traffic.
	A selection process on this underlying topology is essential to our minds.
	
      \item Using one radio on each accesspoint for a common overall-channel seemed to us as a waste of radio moduels.
	Radio modules on accesspoints are scarce and should be utilized efficiently. 
	Furthermore one overall-channel which is utilized by each accesspoint 
	does not scale very well and is basically already AutoWDS basic.
	
      \item The ranking process suggested to run on the accesspoints creates load on the accesspoints.
	We want to use the accesspoints merely as sensors and not as decision-makers, since this creates additional load on the APs.
	
      \item The ranking algorithm itself could be improved. 
	For our taste it averages values too early in the process and uses a random pick not as a last resort.
	
      \item Its \ac{BFS} order of assigning channels puts one node (normally the gateway-node) above all others during channel assignment.
	In some of our usescases there is no gateway node available and we need a fair distribution of channels, which is not feasible with this solution \cite{overview_caa}.
    \end{itemize}

  \section{\ac{CTA}}
    Subramanian et al. use tabu search 
    topology preserving
    usable with commodity 802.11 hardware
    colors the nodes in the conflict graph
      uses tabu search:
	ausgehend von random color assignment in conflict graph
	verstehe nicht wie das weitergeht.
    \cite{CTA}.
  