\chapter{Evaluation}
How do we evaluate the Solution?\newline
  By measuring the overall throughput and comparing the scenarios\newline
\section{Test arrangement}
What was the test-arrangement? \newline
  \subsection{Physical Infrastructure}
    Image of floorplan \newline
    Where was the system placed? \newline
    What are the caracteristics of the physical environment? (Walls/Interference from other devices/...) \newline
    Limitations due to physical restrictions: \newline
      Distance between APs maybe not big enough, despite transmission power reduction\newline
      So the Hidden station Problem not fully visible -> results should be even better compared to base scenario \newline
  \subsection{Network Infrastructure}
    Image of AutoWDSstatus Topology
    OpenVZ \newline
    VLAN \newline
    IPerf \newline
\section{Metrics}
  \subsection{Durations}
    How long were the tests run? \newline
      10 Minutes
    Measurement Intervalls? \newline
      Every 7 Seconds
  \subsection{Channelusage}
    Which Channels were used and in what combinations? \newline
      1 \newline
      36 \newline
      1,6,11 \newline
      36,40,44 \newline
      1,6,11,36,40,44 \newline
  \subsection{Characteristics}
    When did the tests take place?\newline
      Mainly Sundays / Holidays in the evening \newline
    Transmission power \newline
      Antenna gain 3 dB \newline
      Antenna gain 20 dB + Transmitt power reduction \newline
\section{Results}
  \subsection{Expectations}
    What results would we expect?\newline
  \subsection{Actual Results}
    Do the actual results diverge from the expected ones?\newline
    Assessment of the base scenario \newline
      Base Scanario (uses only 1 Channel for all connections) pretty much broken by design -> Medium totally overloaded -> does not even scale to 3 APs in close proximity \newline
    By how much is the solution better then before? \newline
      Despite being actually usable again, Traffic throughput is ~7-10 Times higher
\section{Reflection on the requirements}
  How far does the solution meet the requirements?\newline
  \subsection{Increased Throughput}
    Definitely, see Diagrams \newline
  \subsection{Reduced Connectivity failures}
    Not tested, since test equippment lacks support for multi-flow/routing support (only bridged connections between APs) \newline
  \subsection{Multiple Radios utilized}
    Surely, with excess radios being usable for client connections \newline
  \subsection{Solution works within the perimeter}
    Absolutely, since we can compute CAA on central entity and works best for static scenarios \newline
  \subsection{Comply with Economic Restrictions}
    Runtime for Small scenarios (about 13 APs with about 500 Possible connections) on a modern system (fill in description of system) small -> <1 Second
    Estimate for bigger scenario missing (How do i correctly estimate that?)
\section{Reflection on related Work}
  Comparison to related Work algorithms\newline
  \subsection{Features of other Systems}
  \subsection{Features of our System}
\section{Discussion}
  What do the results mean? \newline
    Obviously it is useful to use mutliple channels for WDS system \newline
  What else questions belong here or is it just the one above?
